\documentclass{ximera}

    \author{Bart Snapp}
    \title{Greetings!}
    \begin{document}
    \begin{abstract}
        Nice to meet you!
    \end{abstract}
    \maketitle


Welcome to OSU's geometry-based mathematical and quantitative reasoning general
education course. Our basic goals in this course are to provide foundational
mathematical
literacy, establish habits of problem solving that will identify gross
errors (pun intended!), establish habits of critical autodidacticism
(what does that even mean?), and finally to have some fun with math. In this 3
credit hour course, we learn by doing. Working on new math problems every class
day. The key to keeping this course \textbf{easy} and \textbf{fun} is to
\textbf{come to class everyday}. It's that simple!


\paragraph{Supplies needed}
\begin{itemize}
  \item You will need a way to \textbf{hand in your assignments online via
          \textit{Carmen} as PDF's}. Writing on paper and using a
        \textbf{scanner}
        (found in university libraries) \textbf{will work.}
  \item A \textbf{laptop} or \textbf{tablet} that has a modern web
        browser and a \textbf{keyboard}, might be helpful.
  \item \textbf{A physical protractor.} These protractors
        \begin{center}\small

          \texttt{https://www.amazon.com/dp/B01LS90074/ref=cm\_sw\_em\_r\_mt\_dp\_kiY9FbJVSDQ2V}
        \end{center}
        will suffice.
\end{itemize}

\end{document}
