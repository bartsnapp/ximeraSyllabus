\documentclass[12pt]{amsart}
\usepackage[cm]{fullpage}
\usepackage{multirow}
\usepackage{microtype}
%% \usepackage{pst-barcode}
%% \usepackage{auto-pst-pdf}
\usepackage[T1]{fontenc}
\usepackage[scaled=1,sfdefault]{FiraSans}
\usepackage{FiraMono}
\renewcommand{\le}{\leqslant}
\renewcommand{\ge}{\geqslant}

\begin{document}
\pagenumbering{gobble}
\title{Math 1118: Geometry and our world}%\\ Autumn 2024} %%%% CHANGE

%%%% PER

%%%% COURSE
\maketitle

\subsection*{General Information}\hfill

%%%% CHANGE PER COURSE
%%%% CHANGE PER COURSE
\begin{tabular}{ll}
  Instructor:   & \\
  %Bart Snapp \\
  Email:        & \\%\texttt{snapp.14@osu.edu}\\
  Phone:        & \\
  Office:       & \\%220 Math Building (614) 292-9414\\
  Office Hours: & \\  %TBD
  %% Deadline to Drop:& Friday, January 31st \\ 
   %%                & Friday, March 20th (W)\\
\end{tabular}

%%%% CHANGE PER COURSE
%%%% CHANGE PER COURSE

\subsection*{Greetings}
Welcome to OSU's geometry-based mathematical and quantitative reasoning general
education course. Our basic goals in this course are to provide foundational
mathematical
literacy, establish habits of problem solving that will identify gross
errors (pun intended!), establish habits of critical autodidacticism
(what does that even mean?), and finally to have some fun with math. In this 3
credit hour course, we learn by doing. Working on new math problems every class
day. The key to keeping this course \textbf{easy} and \textbf{fun} is to
\textbf{come to class everyday}. It's that simple!

\subsection*{General education goals}

Successful students will be able to apply quantitative or logical reasoning
and/or mathematical/statistical methods to understand and solve problems and
will be able to communicate their results.

\paragraph{Expected Learning Outcomes}

Successful students are able to:
\begin{enumerate}
  \item[1.1.] Use logical, mathematical and/or statistical concepts and methods
    to
    represent real-world situations.
  \item[1.2.] Use diverse logical, mathematical and/or statistical approaches,
    technologies and tools to communicate about data symbolically, visually,
    numerically and verbally.
  \item[1.3] Draw appropriate inferences from data based on quantitative
    analysis
    and/or
    logical reasoning.
  \item[1.4.] Make and evaluate important assumptions in estimation, modeling,
    and
    logical argumentation and/or data analysis.
  \item[1.5.] Evaluate social and ethical implications in mathematical and
    quantitative
    reasoning.
\end{enumerate}

\subsection*{Supplies needed}
\begin{itemize}
  \item You will need a way to \textbf{hand in your assignments online via
          \textit{Carmen} as PDF's}. Writing on paper and using a
        \textbf{scanner}
        (found in university libraries) \textbf{will work.}
  \item A \textbf{laptop} or \textbf{tablet} that has a modern web
        browser and a \textbf{keyboard}, might be helpful.
  \item \textbf{A physical protractor.} These protractors
        \begin{center}\small

          \texttt{https://www.amazon.com/dp/B01LS90074/ref=cm\_sw\_em\_r\_mt\_dp\_kiY9FbJVSDQ2V}
        \end{center}
        will suffice.
\end{itemize}

\subsection*{Grading} Your grades will be based on the following:
\begin{itemize}
  \item 60\% for the Geometry Journal.
  \item 40\% for quick quizzes.
\end{itemize}
\begin{center}
  \begin{tabular}{|r l | r l | r l | r l | r l |}\hline
    100\%--93\% & A    & 89\%--87\% & B$+$ & 79\%--77\% & C$+$ & 69\%--67\% &
    D$+$
    \\
    92\%--90\%  & A$-$ & 86\%--83\% & B    & 76\%--73\% & C    & 66\%--60\% & D
    \\
                &      & 82\%--80\% & B$-$ & 72\%--70\% & C$-$ & 59\%--0\%  & E
    \\
    \hline
  \end{tabular}
\end{center}

You can check your grades at anytime using \textit{Carmen.} Check out:
\[
  \texttt{https://carmen.osu.edu/}
\]
As grades are posted, you have exactly \textbf{2 weeks} from the
posting date to notify me concerning any errors or irregularities in
\textit{Carmen}.

\subsection*{Class logistics}
We will follow university guidelines concerning the format of the
course.  \textbf{At the time of writing, this is an in-person course.}
Of course, this may be subject to change.

%% \subsection*{Class on Zoom}
%% Each class day I will host a Zoom meeting at 12:40pm. \textbf{This
%%   meeting will be recorded and posted on \textit{Carmen}.}  The Zoom
%% meeting may be shorter than 55 minutes. Once I am through that
%% content, and you have no further questions, we will dismiss.

\subsection*{Textbook/Geometry Journal}
In this class there is \textbf{no expensive textbook}. Instead, you
will \textbf{write your own textbook}. Each class day, I will ask
three questions. Your job will be to answer and record your answers to
these questions. The answers to these questions will become your
\textbf{Geometry Journal}, your own personal textbook for the course.
When working on the questions in your Geometry Journal, you may ask
for help, and help others. You may use the Internet. However, each of
you must hand in your own work, in your own words.
In particular:
\begin{itemize}
  \item \textbf{USING AI TO PRODUCE SOLUTIONS IS
          ACADEMIC MISCONDUCT.}
  \item \textbf{SHARING (PHYSICALLY, VIA EMAIL, OR PHOTO, ETC) YOUR COMPLETED
          WORK WITH ANOTHER STUDENT IS
          ACADEMIC MISCONDUCT.}
  \item \textbf{USING CONTENT (DIGITAL OR OTHERWISE) IN YOUR WORK THAT IS NOT
          YOUR OWN
          IS ACADEMIC MISCONDUCT.}
\end{itemize}

Each of the three questions will be graded out of 2 points (making
each journal assignment worth 6pts) with
\begin{itemize}
  \item 0pts if an honest attempt was not made at the answer.
  \item 1pt if an honest attempt was made, but the answer is not correct.
  \item 2pts if the answer is perfect.
\end{itemize}

Most Journal entries will be due at 11:59pm the day of the next class,
submitted to \textit{Carmen}. Once you hand in your work, I'll grade
it. \textbf{Then you have until 11:59pm on the next day of class to
  redo your work for full credit.}  I will grade your work at most two
times. I will only grade your work twice if you make the initial due
date. If your work is turned in after the initial due date, I will
grade it at most once. If I am somehow delayed in grading, your
deadlines will be adjusted accordingly.
\begin{center}
  \textbf{All homework must be handed in on \textit{Carmen}.}
\end{center}
\begin{center}
  \textbf{Under no circumstances should you email me
    your homework assignment.}
\end{center}

\subsection*{Quizzes}
We'll have quick quizzes \textit{roughly} every week, usually on
Monday. %They will be administered via \textit{Carmen}. You will be
%able to take the quiz at any time on Monday. \textbf{For the quizzes,
%you may use any \textbf{non-human} resources.} 

%% \subsection*{Discussion Forums}
%% If you have questions outside of the Zoom meeting, I \textbf{expect}
%% you to post these questions on the discussion forums on
%% \textit{Carmen}.
%% \begin{itemize}
%% \item If you have a math question, the best place for you to ask it is
%%   on the discussion forum.
%% \item If you have a question about how the class is run, the best
%%   place for you to ask it is on the discussion forum.
%% \end{itemize}

%% \subsection*{Attendance and participation}
%% Your participation will be evaluated based on your performance on the
%% Geometry Journal and the discussion forums. The only reason to meet
%% synchronously, in-person or via Zoom, would be to directly ask me
%% questions with immediate responses.

\subsection*{Contingency Planning} \textbf{Please email me if you need help
  with technical or logistical issues.}
For additional resources, see \texttt{https://keeplearning.osu.edu/}.

\subsection*{Safety Statement}

Your safety, and the safety of those near you, is more important than
attending this class. Please perform daily health checks, and
self-isolate if you are showing symptoms of Covid-19.  If you are on
campus, the University expects you to be wearing a mask. For more
information, see

\noindent
\texttt{https://wexnermedical.osu.edu/features/coronavirus}.

Moreover, as a student you may experience a range of issues that can
cause barriers to learning, such as strained relationships, increased
anxiety, alcohol/drug problems, feeling down, difficulty concentrating
and/or lack of motivation. These mental health concerns or stressful
events may lead to diminished academic performance or reduce a
student's ability to participate in daily activities. The Ohio State
University offers services to assist you with addressing these and
other concerns you may be experiencing. If you or someone you know are
suffering from any of the aforementioned conditions, you can learn
more about the broad range of confidential mental health services
available on campus via the Office of Student Life's Counseling and
Consultation Service (CCS) by visiting ccs.osu.edu or calling
614-292-5766. CCS is located on the 4th Floor of the Younkin Success
Center and 10th Floor of Lincoln Tower. You can reach an on call
counselor when CCS is closed at 614-292-5766 and 24 hour emergency
help is also available through the 24/7 National Suicide Prevention
Hotline at 1-800-273-TALK or at \texttt{suicidepreventionlifeline.org}.

For additional resources, see

\noindent
\texttt{https://studentlife.osu.edu/articles/we-are-here-for-you}.

\subsection*{Diversity}
The Ohio State University affirms the importance and value of
diversity in the student body. Our programs and curricula reflect our
multicultural society and global economy and seek to provide
opportunities for students to learn more about persons who are
different from them. We are committed to maintaining a community that
recognizes and values the inherent worth and dignity of every person;
fosters sensitivity, understanding, and mutual respect among each
member of our community; and encourages each individual to strive to
reach his or her own potential. Discrimination against any individual
based upon protected status, which is defined as age, color,
disability, gender identity or expression, national origin, race,
religion, sex, sexual orientation, or veteran status, is prohibited.

%% \subsection*{Course Overview} 
%% This course consists of 4 chapters (depending on time):
%% \begin{multicols}{2}
%% \begin{enumerate}
%% \item Isometries
%% \item Folding and Tracing Constructions
%% \columnbreak
%% \item Higher Dimensions
%% \item String Art
%% \end{enumerate}
%% \end{multicols}

\subsection*{Academic Misconduct Statement}
It is the responsibility of the Committee on Academic Misconduct to investigate
or establish procedures for the investigation of all reported cases of student
academic misconduct. The term ``academic misconduct'' includes all forms of
student academic misconduct wherever committed; illustrated by, but not limited
to, cases of plagiarism and dishonest practices in connection with
examinations. Instructors shall report all instances of alleged academic
misconduct to the committee (Faculty Rule 3335-5-48.7 (B)). For additional
information, see the Code of Student Conduct.
\noindent\texttt{https://studentlife.osu.edu/csc/}.

\Huge
\subsection*{Disability Services Statement}
The university strives to maintain a healthy and accessible environment to
support student learning in and out of the classroom.  If you anticipate or
experience academic barriers based on your disability (including mental health,
chronic, or temporary medical conditions), please let me know immediately so
that we can privately discuss options.	To establish reasonable accommodations,
I may request that you register with Student Life Disability Services.	After
registration, make arrangements with me as soon as possible to discuss your
accommodations so that they may be implemented in a timely fashion.

If you are ill and need to miss class, including if you are staying home and
away from others while experiencing symptoms of a viral infection or fever,
please let me know immediately. In cases where illness interacts with an
underlying medical condition, please consult with Student Life Disability
Services to request reasonable accommodations. You can connect with them at

\noindent
\texttt{slds@osu.edu};

\noindent
614-292-3307;

\noindent
\texttt{https://slds.osu.edu};

\noindent 098 Baker Hall, 113
W. 12th Avenue.  \normalsize

\normalsize
\subsection*{Religious accommodations}

Ohio State has had a longstanding practice of making reasonable academic
accommodations for students' religious beliefs and practices in accordance with
applicable law. In 2023, Ohio State updated its practice to align with new
state legislation. Under this new provision, students must be in early
communication with their instructors regarding any known accommodation requests
for religious beliefs and practices, providing notice of specific dates for
which they request alternative accommodations within 14 days after the first
instructional day of the course. Instructors in turn shall not question the
sincerity of a student's religious or spiritual belief system in reviewing such
requests and shall keep requests for accommodations confidential.

With sufficient notice, instructors will provide students with reasonable
alternative accommodations with regard to examinations and other academic
requirements with respect to students' sincerely held religious beliefs and
practices by allowing up to three absences each semester for the student to
attend or participate in religious activities. Examples of religious
accommodations can include, but are not limited to, rescheduling an exam,
altering the time of a student's presentation, allowing make-up assignments to
substitute for missed class work, or flexibility in due dates or research
responsibilities. If concerns arise about a requested accommodation,
instructors are to consult their tenure initiating unit head for assistance.

A student's request for time off shall be provided if the student's sincerely
held religious belief or practice severely affects the student's ability to
take an exam or meet an academic requirement and the student has notified their
instructor, in writing during the first 14 days after the course begins, of the
date of each absence. Although students are required to provide notice within
the first 14 days after a course begins, instructors are strongly encouraged to
work with the student to provide a reasonable accommodation if a request is
made outside the notice period. A student may not be penalized for an absence
approved under this policy.

If students have questions or disputes related to academic accommodations, they
should contact their course instructor, and then their department or college
office. For questions or to report discrimination or harassment based on
religion, individuals should contact the Office of Institutional Equity.

\end{document}
